\documentclass[paper=a4,12pt,titlepage,listof=totoc,index=totoc,bibliography=totoc]{scrartcl}
\usepackage[english]{babel}
\usepackage[T1]{fontenc}
\usepackage[english]{translator}
\usepackage{graphicx}    % zur Einbindung von Bilddateien
\usepackage{scrpage2}    % notwendig um z.B. Kopfzeile zu definieren
\usepackage{subfig}      % mehrere Bilder nebeneinander
\usepackage{psfrag}
\usepackage[numbers]{natbib}   % anstelle von cite, bessere Zitier-Moeglichkeiten
\usepackage{amsmath}
\usepackage[numbers]{natbib}
\usepackage{float}

\begin{document}
\section*{Linear Stability Analysis of CAF}
Hickox \cite{Hickox} was the first to consider the full linear stability analysis of core annular flow (CAF). 
He considered axisymmetric perturbations and perturbations of angular mode $n=1$ and used an asymptotic expansion for long waves allowing him to solve the problem analytically. 
Hickox formulated the problem in a general way but only considered the case where the thin fluid is in the core. 
In 1984 Joseph and Renardy studied the case of equal densities and no surface tension \cite{JRR84}. 
They discussed why the concentrical flow configuration is the one observed in general and the role of the viscous dissipation principle. 
They solved the linear stability problem numerically using Chebyschev-polynomials in radial direction and reported that the situation when the thin fluid at the core is unstable but the thick fluid at core may also be unstable.\par
In a series of papers \cite{SCAF1,SCAF2,SCAF3,SCAF4,SCAF5} starting in 1989 Joseph et Al studied thoroughly the annular flow configuration in the context of lubricated pipelining. 
In the first paper \cite{SCAF1} solved the linear stability problem numerically using a pseudo-spectral method. 
They identified a window of parameters in which core annular flow was stable to small disturbances. 
They found that the wavelengths of maximum growth rate compared well with the size of bubbles in experiments. 
The paper \cite{SCAF2} revisited the linear stability problem giving further insights by considering the energy of the linear perturbations. 
They compared the size of various terms in the integral energy balance and used them to explain the experimental observations of different flow types: wavy annular flow, bubble, drops and emulsifications of oil in water. 
In \cite{SCAF1,SCAF2} the comparison between linear stability analysis and experiments was found to be very favourable. 
This is in contrast to other situations in fluid dynamics, e.g. the transition to turbulence of Poiseuille and Couette Flows, where linear stability analysis does not predict the experimental observations.\par
Joseph et Al. also found the situation with the less viscous fluid in the core to be always unstable and usually waves of finite amplitude were observed. 
These waves were predicted to be unstable by linear theory but could be stabilized by non-linear effects. 
The waves are expected to occur from bifurcation of stable CAF
\par
To check the stability of annular flow, small perturbations are decomposed into Fourier modes:
\begin{equation*}
 \left[ \tilde u, \tilde v, \tilde w, \tilde p, \tilde \delta \right] = \left[ i u(r), v(r), w(r), p(r), \delta \right] exp\left(in\theta+i\alpha (x - c \, t)\right)
\end{equation*}
The linearized axial component momentum equation reads
\begin{equation*}
 \alpha (W - c) w + W' u = -\alpha p - \frac{i}{Re_l} \left(w''+\frac{w'}{r}-(\alpha^2+\frac{n^2}{r^2})w \right)
\end{equation*}
and the continuity equation:
\begin{equation*}
 u'+\frac{u}{r}+\frac{nv}{r}+\alpha w=0.
\end{equation*}
The boundary conditions are
\begin{equation*}
 u(a)=v(a)=w(a)=0,
\end{equation*}
at the pipe wall and
\begin{eqnarray*}
 n\ne 0: \\
 \frac{d v}{d\theta} = \frac{d }{d\theta} (\tilde w e_x + \tilde u e_r + \tilde v e_{\theta}) =0 \quad \quad (r=0)\\
 \Rightarrow n i w(0) e_x - ( n u(0) + v(0) )e_r + i (n v(0) + u(0) ) e_{\theta}=0\\
  n=0: \\
 u(0)=v(0)=\alpha w(0) + 2 u'(0)
\end{eqnarray*}
on the symmetry axis. 
At the interface the kinematic BC:
\begin{eqnarray*}
 [u]=[v]=0, \\
 \left[W'\right] u(1) + \alpha (W-c)[w]=0,
\end{eqnarray*}
and dynamic BC:
\begin{eqnarray*}
 \left[ m_l(w'-\alpha u) \right] &=& 0, \\
 \left[ m_l(v'-v-nu) \right] &=& 0, \\
 -\left[ \zeta_l p \right]+2 i \left[m_l u'\right] &=& \frac{J}{Re_1^2} (1-\alpha^2-n^2) \frac{u(1)}{\alpha(W-c)},
\end{eqnarray*}
have to hold. \par

\subsection*{Energy Analysis}

The energy analysis is useful to determine which physical mechanism is causing the instability. 
The energy balance of linear perturbations to CAF was successfully used by \cite{SCAF2} to identify three destabilizing mechanisms namely: surface tension, interfacial friction and Reynolds stress. 
While for a single fluid, the energy integral only contains bulk contributions for multiple fluids additional interfacial terms appear. 
These provide the energy method for several fluids with the potential to uncover destabilizing mechanisms in a way that was not possible for one fluid.\par
Starting point of the energy analysis are the linearized equations for the evolution of the perturbations. 
After multiplying the linearized momentum equations by the complex conjugate of the eigenfunctions, integrating over each subdomain and adding together, one obtains an equation whose imaginary part is
\begin{eqnarray*}
\alpha  c_i a \sum \limits_{l=1}^2 \zeta_l \int_{\Omega_l}  |\vect{u}|^2 r dr = a \sum \limits_{l=1}^2 \zeta_l \int_{\Omega_l}  W' Imag(u \bar{w}) r dr \\
-\frac{1}{Re} \sum \limits_{l=1}^2 m_l \int_{\Omega_l} \left( \bigg|\frac{d(r u)}{r dr}\bigg|^2+\bigg|\frac{d(r v)}{r dr}\bigg|^2+\bigg|\frac{dw}{dr}\bigg|^2 + (\alpha^2 +\frac{n^2}{r^2}) |\vec{u}|^2 + \frac{4n}{r^2} Real(u\bar{v})\right) r dr \\
- \frac{|u(0)|^2+|v(0)|^2}{Re} \\
+ c_i a \frac{J}{\alpha Re^2} \frac{(1.0 - k^2 - n^2}{|W-c|^2} |u(1)|^2 \\
+ \frac{1}{Re} Real\left(( |u(1)|^2+|v(1)|^2 ) [m] - \bar{u(1)} [m u'] + \bar{v(1)} [m v'] + [m \bar{w} w'] \right), \\
\mbox{where} \quad |\vect{u}|^2 = |u|^2+|v|^2+|w|^2.
\end{eqnarray*}
Since the amplitude of the velocities $u,v,w$ is arbitrary, following \cite{SCAF2} they are normalized such that $D=1$. 
...
All these terms can have positive or negative values except for...
and they compete to determine if the total kinetic energy of a disturbance will increase or decrease. 
\par
Interfacial tension is always dominant and destabilizing for small Reynolds numbers. 
Interfacial friction can stabilize interfacial tension but it is the dominant mechanism destabilizing the flow when the thin fluid in the core. 
The Reynolds stress in the core is not destabilizing but in the annulus it can lead to instability in both flow configurations. 

\subsection*{Jet Limit}

Dispertion relation:
\begin{equation*}
 \omega = \sqrt{ k  (k^2 - 1) \frac{I_1(k)}{I_0(k)} }.
\end{equation*}
Parameters:
\begin{eqnarray*}
 m \rightarrow 0, \quad \zeta \rightarrow 0, \quad \frac{J}{Re}\rightarrow \infty \\
 m = 10^{-3}, \quad \zeta = 10^{-4}, \quad J=10^{10}, \quad Re_1=10^{-3} \\ a=1.25
\end{eqnarray*}

\subsection*{Discretization with Chebyshev}
\begin{equation*}
 T_k (x) = cos(k \, arccos(x))
\end{equation*}

Colocation pts:
\begin{equation*}
 x_j=cos\frac{\pi j}{N},\quad j=0,1,...,N \quad x_j\in[-1,1]
\end{equation*}
Interpolant:
\begin{equation*}
 p_N^{(n)}(x)=\sum_{j=1}^N \phi_j^{(n)}(x) f_j.
\end{equation*}

\begin{eqnarray*}
 \phi_j(x) = \frac{(-1)^j}{c_j} \frac{1-x^2}{N^2} \frac{T_N ' (x)}{x-x_j}, \\
 \phi_j(x) = \prod_{k=1,k\ne j}^N \frac{x-x_k}{x_j-x_k},
\end{eqnarray*}
derivative matrix:
\begin{equation*}
 f^{(n)}_i \approx \sum \limits_{j=1}^N \phi_j^{(n)}(x_i) f_j.
\end{equation*}

\begin{thebibliography}{1}
\bibitem{SCAF1}\textbf{Preziosi, L.; Chen, K.; Joseph, D.} \textsl{Lubricated pipelining: Stability of core-annular flow.} Journal of Fluid Mechanics \textbf{201}: 323-356, 1989
\bibitem{SCAF2}\textbf{Hu, H.; Joseph, D.} \textsl{Lubricated pipelining: Stability of core-annular flow. Part 2.} Journal of Fluid Mechanics \textbf{205}: 359-396, 1989.
\bibitem{SCAF3}\textbf{Chen, K.; Bai, R.; Joseph, D.} \textsl{Lubricated pipelining: Stability of core-annular flow. Part 3. Stability of core-annular flow in vertical pipes.} Journal of Fluid Mechanics \textbf{214}: 251-286, 1990.
\bibitem{SCAF4}\textbf{Chen, K.; Joseph, D.} \textsl{Lubricated pipelining: Stability of core-annular flow. Part 4. Ginzburg–Landau equations.} Journal of Fluid Mechanics \textbf{227}: 587-615, 1991..
\bibitem{SCAF5}\textbf{Bai, R.; Chen, K.; Joseph, D.} \textsl{Lubricated pipelining: Stability of core-annular flow. Part 5. Experiments and comparison with theory.} Journal of Fluid Mechanics \textbf{240}: 97-132, 1992.
\bibitem{Hickox}\textbf{Hickox, C.E.} \textsl{Instability due to Viscosity and Density Stratification in Axisymmetric Pipe Flow} Phys. Fluids \textbf{14}: 251-262, 1971.
\bibitem{JRR84}\textbf{Joseph, D.; Renardy, M.; Renardy, Y.}\textsl{Instability of the flow of two immiscible liquids with different viscosities in a pipe.} Journal of Fluid Mechanics \textbf{141}: 309-317, 1984.
\end{thebibliography}
\end{document}
